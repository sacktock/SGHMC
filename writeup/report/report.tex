%!TEX options = -shell-escape

\documentclass[a4paper]{article}

% Suppress some useless warnings
\usepackage{silence}
\WarningFilter{remreset}{The remreset package}
\WarningFilter{latex}{Font shape declaration has incorrect series value `mc'.}

\usepackage[UKenglish]{babel}
\usepackage[style=alphabetic,maxbibnames=99]{biblatex}
\usepackage{amsmath}
\usepackage{mathtools}
\usepackage{amsthm}
\usepackage{thmtools}
\usepackage{csquotes}
\usepackage{commath}
\usepackage{url}
\usepackage{algorithm}
\usepackage{algpseudocode}
\usepackage[inline]{enumitem}
\usepackage[hidelinks]{hyperref}
\usepackage[capitalize,noabbrev]{cleveref}
\usepackage[final]{microtype}
\usepackage{caption}
\usepackage{subcaption}

\usepackage[charter]{mathdesign}
\usepackage[T1]{fontenc}

% Fix me notes
\usepackage[draft]{fixme}
\usepackage{svg}
\fxusetheme{color}

% Theorem styles
\newtheorem{theorem}{Theorem}
\newtheorem{lemma}[theorem]{Lemma}
\newtheorem{proposition}[theorem]{Proposition}
\newtheorem{corollary}[theorem]{Corollary}
\newtheorem{question}[theorem]{Question}
\newtheorem*{mainquestion}{Main question}
\theoremstyle{definition}
\newtheorem{definition}[theorem]{Definition}
\newtheorem{construction}[theorem]{Construction}
\newtheorem{example}[theorem]{Example}
\theoremstyle{remark}
\newtheorem{remark}[theorem]{Remark}

% LaTeX shorthands
% Single letters
\newcommand{\R}{\ensuremath{\mathbb{R}}}
\newcommand{\Q}{\ensuremath{\mathbb{Q}}}
\newcommand{\Z}{\ensuremath{\mathbb{Z}}}
\newcommand{\N}{\ensuremath{\mathbb{N}}}
\newcommand{\D}{\ensuremath{\mathcal{D}}}

% Shorthands
\newcommand{\ep}{\epsilon}
\newcommand{\lam}{\lambda}
\newcommand{\Lam}{\Lambda}
\newcommand{\la}{\leftarrow}
\newcommand{\ra}{\rightarrow}
\newcommand{\lra}{\leftrightarrow}
\newcommand{\La}{\Leftarrow}
\newcommand{\Ra}{\Rightarrow}
\newcommand{\Lra}{\Leftrightarrow}
\newcommand{\es}{\varnothing}
\newcommand{\sse}{\subseteq}
\newcommand{\partto}{\rightharpoonup}
\newcommand{\ol}{\overline}
\newcommand{\wt}{\widetilde}
\newcommand{\wh}{\widehat}
\newcommand{\bb}{\mathbb}
\newcommand{\mc}{\mathcal}
\newcommand{\mr}{\mathrm}
\newcommand{\defeq}{\vcentcolon=}
\DeclarePairedDelimiter{\ab}{\langle}{\rangle}
\let\abs\undefined
\DeclarePairedDelimiter{\abs}{|}{|}

% Slanted symbols
\renewcommand{\leq}{\leqslant}
\renewcommand{\geq}{\geqslant}
\renewcommand{\nleq}{\nleqslant}
\renewcommand{\ngeq}{\ngeqslant}

% Bibliography
\addbibresource{../references.bib}

\title{Reproducing the paper:\\
\textit{Stochastic Gradient Hamiltonian Monte Carlo}\\
by Tianqi Chen, Emily B. Fox and Carlos Guestrin}
\author{Candidate Numbers: Sam Adam-Day, 1059459, 1060482 and 1058141}
\date{}

\begin{document}

	\maketitle

	\abstract{In this report we reproduce the Stochastic Gradient Hamiltonian Monte Carlo (SGHMC) algorithm contained in \citetitle{sghmc} \cite{sghmc} by \citeauthor{sghmc}. We reproduce a number of the experiments in this paper, and we explain a number of their results. On top of this, we extend their work in a number of ways. We attempt to estimate the value of the hyperparameter $\hat{B}$ in their paper, and we test the algorithm on more than just the MNIST dataset. We also introduce a novel algorithm which we have named SGNUTS, based on the 'No U-Turn Sampler' (NUTS) introduced in \cite{nuts}.}


	%!TEX root = ../report.tex

\section{Introduction}

Hamiltonian Monte Carlo (HMC) provides us with a useful way to sample from a posterior distribution. HMC is an MCMC sampling method that uses all the available data at each step, and it requires a potential function of the form:
\begin{equation*}
    U(\theta) = - \sum_{x\in\mathcal{D}}\log p(x| \theta) - \log p(\theta ) \propto -\log p(\theta | \mathcal{D})
\end{equation*}
which, along with $\nabla U(\theta)$, is sufficient to sample from the posterior (as will be discussed in the background section). Computing $U(\theta)$ and $\nabla U(\theta)$ can be computationally expensive for large datasets, which has motivated the development of Stochastic Gradient Hamiltonian Monte Carlo (SGHMC), first introduced by the paper in question \cite{sghmc}. SGHMC uses randomly sampled mini-batches of data to produce noisy estimates of the gradient, which are denoted by $\nabla \widetilde U(\theta)$. We decided to investigate this paper because it convinced us that SGHMC is a strong candidate for scalable Bayesian inference and it would be interesting to investigate how it compares to more popular methods such as Variational Inference. In this paper, the authors start with a description of HMC, and then introduce Naïve SGHMC algorithm, demonstrating the pitfalls of using noisy gradient estimates. They then go on to introduce the full SGHMC algorithm which uses friction to overcome the need for a costly MH correction step. They then run a number of experiments using SGHMC to empirically back up their theoretical claims and show that SGHMC is a candidate algorithm for scalable Bayesian inference.

We were further motivated to choose this paper as SGHMC is a relatively simple algorithm, and so we would have more time to investigate other datasets and directions.  Another consideration was that running these experiments wouldn't be very computationally demanding, allowing us to run many experiments on our own machines. Below we detail exactly which experiments from \cite{sghmc} we decided to reproduce:

\begin{itemize}
    \item Sampling $\theta$ using the potential function $U(\theta) = -2\theta^2 + \theta^4$, with $\mathcal{N}(0,4)$ noise added to the gradient of this to give $\nabla \widetilde U (\theta)$. This noise potential is a proxy for the noisy potential used by SGHMC. We used the following algorithms: HMC (with and without MH correction), Naive SGHMC (with and without MH correction) and SGHMC.

    \item  Sampling $(\theta,r)$ using the potential function $U(\theta) = \frac{1}{2}\theta^2$, , with $\mathcal{N}(0,4)$ noise added to the gradient of this to give $\nabla \widetilde U (\theta)$. We used the following algorithms: HMC, Naive SGHMC (with and without momentum resampling) and SGHMC.
  \item Comparing the autocorrelation times, as well as the error in the covariance of the samples, of SGHMC and SGLD.
    \item Classifying the MNIST dataset \cite{mnist} using SGHMC as well as with Stochastic Gradient Descent (SGD), Stochastic Gradient Descent (SGD) with momentum, and Stochastic Gradient Langevin Dynamics (SGLD).
\end{itemize}

We also considered some new ideas:

\begin{itemize}
    \item We extended the `No U-Turn Sampler' (NUTS) from \cite{nuts} to work with SGHMC to produce our novel algorithm SGNUTS.
    \item We ran the Bayesian neural network (BNN) to classify a new dataset, namely, FashionMNIST. \cite{fashion-mnist}.
    \item We demonstrated that our implementation of SGHMC can be used with Convolutional Neural Networks (CNNs) to classify CIFAR10 \cite{cifar10}.
    \item We implemented a scheme for estimating the gradient noise ($B$ in the literature and in what follows) and used this to increase the algorithm’s sampling accuracy.
\end{itemize}

The repository for our code can be found at \url{https://github.com/sacktock/SGHMC}.

	%!TEX root = ../report.tex

\section{Background}
	%!TEX root = ../report.tex

\section{Implementation Details}

\begin{itemize}
    \item Describe choice made and how we implemented the algorithms
\end{itemize}
	%!TEX root = ../report.tex

\section{Experiments}

\begin{itemize}
    \item Describe experiments and compare with results in the paper.
\end{itemize}

\subsection{Simulated examples}

\subsection{Bayesian Neural Networks for Classification}
For the Bayesian neural network (BNN) MNIST classification we actually used the reparameterisation of SGHMC described in the section "Connection to SGD with Momentum" of \cite{sghmc}. The SGHMC algorithm is reframed in terms of learning rate and momentum decay, and simulates the following dynamics instead,
$$\begin{cases}
\nabla \theta = v\\
\nabla v = - \eta \nabla \tilde{U}(\theta) - \alpha v + \mathcal{N}(0, 2(\alpha - \hat{\beta}) \eta)
\end{cases}
$$
Where $\eta = \epsilon^2 M^{-1}, \alpha = \epsilon M^{-1}C, \hat{\beta} = \eta M^{-1}\hat{B}$. In all our experiments in this section we set the mass matrix $M $ to the identity, and the noise model $\hat{\beta} = \hat{B} = 0$. Other than the architecture of the BNN the only 3 hyperparameters for SGHMC are the learning rate $\eta$, the momentum decay $\alpha$ and the batch size $|\tilde{D}|$, in all our experiments we fixed $|\tilde{D}| = 500$ which follows what the author's of \cite{sghmc} did. 

To build on top of the work in \cite{sghmc} we implemented learning rate annealing for SGLD and following \cite{sgld} we weighted the samples by the learning rate as follows,
$$\mathbb{E}[f(\theta)] \approx \frac{\sum^T_{t=1} \epsilon_t f(\theta_t)}{\sum^T_{t=1} \epsilon_t}$$
Although since $f$ is a classifier we can ignore the demonimator. Our BNN followed the same architecture as in \cite{sghmc}, that is one linear with 100 hidden units followed by ReLU activation followed by another linear layer and a log softmax for multi-class classification. The weights and biases for both linear layers are sampled from univariate standard normal distributions, but the Pyro method \texttt{to\_event()} declares dependence between the parameters. The likelihood function is of course a Categorical distribution.

Our implementations of SGD and SGD with momentum are meant to be used directly with Pyro BNN, and so Guassian priors on the weights and biases is equivalent to L2 regularization in the non-Bayesian paradigm. We experimented with regularization strengths of $\lambda \in \{0.1, 1.0, 10.0\}$ and found $\lambda = 1.0$ to be the most effective. Additionally we implemented weight decay for both SGD and SGD with momentum but found that this didn't improve anything in this setting.

For the momentum based algorithms, SGHMC and SGD with momentum, we tried $\eta \in \{1.0, 2.0, 4.0, 8.0 \} \times 10^{-6}$, and $\alpha \in \{0.1, 0.01, 0.001 \}$. For SGHMC the best configuration was $\eta = 2.0\time10^{-6}, \alpha=0.01$, and for SGD with momentum the best configuration was $\eta = 1.0\time10^{-6}, \alpha=0.01$.

For SGLD and SGD, we tried $\eta \in \{1.0, 2.0, 4.0, 6.0\} \times 10^{-5}$, for SGLD we also tried learning rate annealing but it proved not to make much of a difference in this setting so we ignored it in the end. The best configuartion for SGLD was $\eta = 4.0\time10^{-5}$, and for SGD the best configuration was $\eta = 1.0\times10^{-5}$.

For MNIST we ran each of the algorithms for 800 epochs with 50 warmup epochs. For the sampling algorithms we do Bayesian averaging over all the sampled parameterisations of the BNN after warmup as is described in Section II of \cite{hands-on-bnn} and report the test error. For the optimization algorithms we take most recent sample / set of parameters as a point estimate and report the test error.  Figure 1 presents our results.

% plot from paper


The results we get from MNIST classification align very closely with those in \cite{sghmc} and so we come to the same conclusion; the need for scalable and efficient Bayesian inference algorithms. The key benefit of BNNs is that we are not overconfident on out-of-distribution examples, Figure 2 illustrates that we still maintain this property when using SGHMC to approximately sample from the posterior distribution. We additionally conducted a brief comparison between Variational Inference (VI) and SGHMC in this setting Figure 3 outlines our findings.

% uncertainty estimates on out-of-distribution
% comparison with VI and discussion
The initial results suggest that SGHMC performs better than VI in this setting, although this is not the full picture. Once VI fits the variational posterior distribution $q_{\pi}$ as closely as possible to the true posterior it takes only hundreds of sample to characterise $q_{\pi}$, whereas SGHMC requires several more samples to characterise the true posterior. In practice storing thousands of parameterisations of the same NN is very costly and so this is probably why VI is a more popular choice for Bayesian inference. We conclude this section by demonstrating our algorithms can be applied to other datasets and more complicated models, Figure 4 presents the results of running the same BNN architecture on fashionMNIST, and Figure 5 demonstrates that our implementation of SGHMC can be used with convolutional neural networks (CNNs).
% other datasets fahsionMNIST, CIFAR10
	\section{NUTS}
\subsection{The Basic Algorithm}

In our current description of the SGHMC algorithm we have the user-defined hyperparameter $m$, the number of steps iterated over before we take a sample. If this number is too small, our samples will be correlated, and hence successive samples would appear to follow a random walk, and we would get slow mixing times. We demonstrate this behaviour by training 3 versions of SGHMC with $m=1,3,5$ (with $\epsilon m$ fixed). The learning curves below in figure \ref{num_steps} show that increasing $m$ increases the speed with which SGHMC reaches low error rates.

\begin{figure}[h!]
\centering
\includegraphics[width=100mm]{parts/Images/changing_num_steps.png}
\caption{Changing the number of steps per sample. Each agent was run for 100 warmup epochs.}
\label{num_steps}
\end{figure}

However, if $m$ is too large we waste computational power, as we continue to step through time even though each sample is already independent of the last. We would like to set $m$ to its optimal value.

Hoffman and Gelman introduce the algorithm NUTS in their paper ``The No-U-Turn Sampler: Adaptively Setting Path Lengths in Hamiltonian Monte Carlo'' \cite{nuts}. In its basic form this algorithm removes the need for a user to input a value for $m$ in the standard HMC implementation. We converted this NUTS algorithm from being based on HMC to being based on SGHMC, and investigated its power. We will begin by presenting the high level overview of the original NUTS algorithm for HMC sampling.

The key idea is the concept of a `U-Turn'. This is the point at which the samples of $\theta$ start to get closer to the initial value of $\theta$, rather than away from it. This marks the point at which further steps will likely only waste computational power. Mathematically, for current position $\theta$, initial position $\theta_0$ and momentum $r$, this corresponds to the time at which:
$$ \frac{d}{dt} |\theta - \theta_0|^2 = 2(\theta - \theta_0)\cdot\frac{d\theta}{dt} = 2(\theta - \theta_0)\cdot r< 0$$
where we use the fact that in the dynamics of HMC (and also SGHMC) we have $\frac{d\theta}{dt} = r$. This simple fact suggests an algorithm in which we draw the sample $\theta$ once we have performed enough steps so that $(\theta - \theta_0)\cdot r< 0$. However this is too simplistic an approach - as HMC is an instance of the Metropolis Hastings algorithm we require the Markov Chain of $\theta$ to be reversible, which is not the case here.

To remedy this problem, NUTS requires keeping track of a set $\mathcal{B}$, which contains all values of $(\theta, r)$ as steps are taken both forward and backwards in time. The values at the earliest and latest times considered across a single trajectory are labelled  $(\theta^-, r^-)$ and $(\theta^+, r^+)$ respectively. NUTS starts from a single $(\theta, r)$ and then steps either forward or backwards one step. It then steps forward or backwards 2 steps, then 4 steps, then 8 steps etc.\@ until a `U-Turn' is seen. See \cref{alg:nuts}.

\begin{algorithm}
    \caption{The NUTS algorithm}\label{alg:nuts}
    \begin{algorithmic}[5]
        \Require $(\theta_0, r_0)$
        \State $r \sim \mc N(0,1)$
        \State $n \gets 1$
        \State $\mathcal{B}\gets \{(\theta_0, r_0)\}$
        \State $(\theta^-, r^-) \gets (\theta_0, r_0) $
        \State $(\theta^+, r^+) \gets (\theta_0, r_0) $
        \While{there is no U-Turn at $(\theta^-, r^-)$ nor at $(\theta^+, r^+)$ (ie $(\theta^+ - \theta^-)\cdot r^- \geq 0$ and $(\theta^+ - \theta^-)\cdot r^+ \geq 0$)}
            \State With probability $\frac 1 2$, choose to go \textit{forwards} or \textit{backwards} in time
            \If{\textit{forwards}}
                \State $(\theta, r) \gets (\theta^+, r^+)$
                \For{$i = 1$ to $n$}
                   \State {$(\theta, r) \gets $ step forward in time from $(\theta, r)$} \label{line:step forward; nuts}
                   \State $\mathcal{B} \gets \{(\theta, r)\} \cup \mathcal{B}$
                \EndFor
                \State $(\theta^+, r^+) \gets (\theta, r)$
            \EndIf
            \If{\textit{backwards}}
                \State $(\theta, r) \gets (\theta^-, r^-)$
                \For{$i = 1$ to $n$}
                    \State {$(\theta, r) \gets $ step backwards in time from $(\theta, r)$}\label{line:step back; nuts}
                    \State $\mathcal{B} \gets \{(\theta, r)\} \cup \mathcal{B}$
                \EndFor
                \State $(\theta^-, r^-) \gets (\theta, r)$
            \EndIf
            \State $n \gets 2n$
        \EndWhile
        \State{Carefully choose a subset $\mathcal{C} \subseteq \mathcal{B}$}\label{line:choose subset; nuts} \Comment{\emph{This step is the key to the Markov Chain being reversible; we don't go into detail here}}
        \State{Sample an element of $\mathcal{C}$}
    \end{algorithmic}
\end{algorithm}

The benefit to this algorithm is that it removes the need to set the number of steps performed before we sample, as we keep stepping until a `U-Turn' is seen. We should note here that in the original form of NUTS, the `step' being referred to in \cref{line:step forward; nuts} and \cref{line:step back; nuts} is the step of the HMC algorithm. We edited the NUTS Pyro source code to make it perform SGHMC steps, and we named this SGNUTS (Stochastic Gradient No U-Turn Sampler).

There were some doubts as to whether the NUTS algorithm would work when using SGHMC steps instead of HMC steps. This was because NUTS requires the ability to step backwards in time, while an SGHMC step includes an injection of stochastic noise. As there is no action that undoes this injection, there was a worry that the backwards step through time in NUTS would become a problem. We explain how we attempted to solve this problem at the end of the next section.


\subsection{Implementation of SGNUTS}

To build SGNUTS we started with the Pyro source code for NUTS \cite{nuts_code} which takes the Pyro HMC class as its parent. We altered this to instead inherit from our SGHMC class. This required removing step-size adaptation and mass matrix adaptation functionality from NUTS, as our implementation of SGHMC was not able to interface with this. It also required introducing some caching methods into the SGHMC class - to help keep things simple in our original SGHMC class we did this in a new class, named SGHMC\_for\_NUTS. Most importantly, we had to alter the $(\theta,r)$ step update rule which is hardcoded in NUTS. This meant that instead of being the HMC update step it was now the SGHMC update step of:

$$
    \theta \leftarrow 
    \begin{cases}\theta + \epsilon M^{-1}r, & \text{{\textit{forwards step}}}\\
                \theta - \epsilon M^{-1}r,  & \text{{\textit{backwards step}}}

\end{cases}
$$
$$
    r \leftarrow 
    \begin{cases}
                
    r - \epsilon \nabla \tilde{U}(\theta) - \epsilon C M^{-1}r + \mathcal{N}(0,2(C-\hat{B})\epsilon), & \text{{\textit{forwards step}}}\\
   
    r + \epsilon \nabla \tilde{U}(\theta) - \epsilon C M^{-1}r + \mathcal{N}(0,2(C-\hat{B})\epsilon), & \text{{\textit{backwards step}}}\\
 \end{cases}
    $$

In particular, note that there is an injection of stochastic noise and momentum-reducing friction in both the forward steps and the backward steps. This is not ideal, as our backwards step in time does not undo a forward step, which likely breaks the reversibility of the Markov Chain being considered in NUTS. However we investigated this nonetheless.


\subsection{Results} 

We tested the SGNUTS algorithm on the BNNs described earlier. We ran SGNUTS for only 20 warmup epoch and 50 epochs as the code was slower than SGHMC to run. The learning curves were as shown in \cref{SGNUTS_results}.

\begin{figure}[h!]
\centering
\includegraphics[width=70mm]{parts/Images/SGNUTS_MNIST.png}
\includegraphics[width=75mm]{parts/Images/SGNUTS_Fashion.png}
\caption{Classifying MNIST and FashionMNIST with SGNUTS}
\label{SGNUTS_results}
\end{figure}

The final accuracies were 0.94 for MNIST and 0.85 for FashionMNIST which are similar to the accuracies obtained using SGHMC (0.97 and 0.86 respectively). This accuracy demonstrates that there is convergence to the true posterior in SGNUTS. This suggests that, like SGHMC itself, the SGNUTS Markov Chain is likely reversible in some (non traditional) sense. Denoting the posterior as $\pi(\theta, r)$, and transition kernels as $P(\theta, r| \theta', r')$, it is shown in \cite{sghmc} that SGHMC satisfies:
$$ \pi(\theta,r)P_{SGHMC}(\theta,r|\theta',r') = \pi(\theta',-r')P_{SGHMC}(\theta',-r'|\theta,-r)$$
and it is suggested that this is the reason why the SGHMC algorithm works, despite not being reversible in the traditional sense. It would be interesting to consider if a similar property holds for $P_{SGNUTS}$.

While accurate, SGNUTS is slower to run than SGHMC to produce a single sample. It does however reach relatively high accuracies in a short amount of time. We measured the accuracy after the first epoch of SGHMC and NUTS as we changed the number of warmup epochs, and we measured how long they took to train.

\begin{center}
\begin{tabular}{ccccc} 
\hline
Number of Warmup-Epochs &\multicolumn{2}{c}{SGHMC}& \multicolumn{2}{c}{SGNUTS} \\
& Accuracy & Time (s) & Accuracy & Time (s) \\
 \hline
 0 &0.7747  &  3.2&0.8715&28.3\\ 
 1 & 0.5764 & 4.5 &0.9296&127.2\\ 
 2 & 0.4141 & 6.2&0.9231&453.3\\ 
 5 &0.6471& 11.4 &&\\ 
 10 &0.7100& 20.3 &&\\ 
 25 &0.8866& 50.2 &&\\ 
 \hline
\end{tabular}
\end{center}

In particular, SGNUTS reached an accuracy of 0.87 in 28s, while SGHMC reached 0.89 in 50s. This suggests a new algorithm that uses both SGNUTS and SGHMC — we could run a single epoch of SGNUTS so that we quickly approach the posterior distribution, at which point we switch to running SGHMC. It should be noted the speed up suggested by the above results would only be 30s seconds, however maybe on more complex datasets this could be larger. It would be interesting to investigate this if we had more time.
	%!TEX root = ../report.tex

\section{Conclusion}

For this project, we have implemented a number of algorithms for the purpose of replicating and extending the experiments in \cite{sghmc}. We opted for a close integration with Pyro which allowed us to make use of its probabilistic programming framework. 

Replicating the analysis in \cite{sghmc}, we tested SGHMC on a BNN for classifying MNIST digits, comparing it to SGLD, SGD and SGD with momentum. We obtained similar results to the original experiment. Moreover, we expanded the analysis by testing SGMHC on a BNN for the Fashion-MNIST dataset and a CNN for CIFAR10. In both cases SGMHC performed very well. In the former we further compared SGLD, SGD and SGD with momentum, with SGMHC producing the superior performance. Given the close integration of our codebase with Pyro, it is easy to extend to further models and datasets, which we would do if given more time.

As a further extension, we briefly compared SGHMC with the more popular variational inference method, using our BNN on the MNIST dataset. Over a large number of epochs SGHMC performs better than VI. However, the latter converges faster, and requires fewer samples to give a representative picture of the posterior. With more time, we would like to investigate this further, for example by comparing the sampling efficiency more in-depth.

Our implementation of SGHMC provided the option of estimating the noise model using the observed information, as suggested in \cite{sghmc}. (More details based on results of experiments). Given more time, we would like to investigate alternative methods of estimating the noise model, ideally ones less computationally expensive.

Our last extension to \cite{sghmc} was the specification and implementation of the model method SGNUTS, which combines SGHMC with the No-U-Turn Sampler. In spite of the lack of reversibility of the corresponding Markov process, our algorithm performs well on a BNN for the MNIST and Fashion-MNIST datasets. We found that while SGNUTS takes longer than SGHMC to produce a single sample, but reaches high accuracies in a relatively small number of steps. If we were to continue this project, we would investigate in more detail the theoretical and practical aspects of SGNUTS.

	\printbibliography


\end{document}