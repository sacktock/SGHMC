%!TEX root = ../report.tex

\section{Background}

\subsection{HMC}

Hamiltonian Monte Carlo (\cite{duane-hmc,neal-hmc}) is a Markov Chain Monte Carlo (MCMC) sampling algorithm. Given a target probability distribution — in our case the posterior distribution of a set of variables $\theta$ given independent observations $x \in \D$ — it produces samples by carrying out a random walk over the parameter space using Hamiltonian dynamics.

We begin with the prior distribution $p(\theta)$ and likelihood $p(x \mid \theta)$. Using these we define the \emph{potential energy} function $U$:
\begin{equation*}
    U(\theta) \defeq - \sum_{x \in \D}\log p(x \mid \theta) - \log p(\theta)
\end{equation*}
Note that, using Bayes' rule, we have that the posterior $p(\theta \mid \D) \propto \exp(-U)$. Hamiltonian dynamics introduces an auxiliary set of momentum variables $r$. These dynamics have a physical interpretation in which an object moves about a landscape determined by $U$. We let this object have \emph{mass matrix} $M$, which is typically set to the identity matrix. Then $U(\theta)$ represents the potential energy of the object, and its kinetic energy is given by $\frac 1 2 r^{\mathsf T} M^{-1} r$. The total energy of the system is a quantity known as the \emph{Hamiltonian function}:
\begin{equation*}
    H(\theta, r) = U(\theta) + \frac 1 2 r^{\mathsf T} M^{-1} r
\end{equation*}
The development of the system is governed by the following equations.
\begin{align*}
    \dif\theta &= M^{-1}r \dif t \\
    \dif r &= - \nabla U(\theta) \dif t
\end{align*}

To simulate these continuous dynamics in practice, we must use a discretised version of these equations. To correct for the inaccuracies introduced by doing so, it is necessary to make a \emph{Metropolis-Hastings correction step}. A simple algorithm is given in \cref{alg:hmc}.

\begin{algorithm}
    \caption{A simple HMC algorithm}\label{alg:hmc}
    \begin{algorithmic}
        \For{$t = 1,2, \ldots$}
            \State $r \sim \mc N(0,1)$ \Comment{Resample momentum}
            \State $(\theta_0, r_0) = (\theta, r)$ 
            \For{$i = 1$ to $m$}
                \State $\theta \gets \theta + \ep M^{-1}r$
                \State $r \gets r - \ep \nabla U(\theta)$
            \EndFor
            \State $u \sim \rm{Uniform}[0,1]$
            \State $\rho = \exp(H(\theta, r) - H(\theta_0, r_0))$ \Comment{Acceptance probability}
            \If{$u < \min(1, \rho)$} \Comment{Only accept new state with probability $\rho$}
                \State $\theta = \theta_0$
            \EndIf
        \EndFor
    \end{algorithmic}
\end{algorithm}

In practice, the dataset $\D$ may be large, and so running \cref{alg:hmc} may be computationally expensive, as the complexity of calculating $\nabla U$ scales with $\abs\D$. One idea to combat this is to simulate the Hamiltonian system using only a subset of the data at a time, in analogy with stochastic gradient descent. This method is known as Stochastic Gradient Hamiltonian Monte Carlo (SGHMC), however before explaining the SGHMC algorithm we begin with Naïve SGHMC.


\subsection{Naïve SGHMC}

To understand the Naïve SGHMC method, consider sampling a minibatch $\wt\D \subset \D$ uniformly at random. We estimate the gradient $\nabla U(\theta)$ using this minibatch as follows:
\begin{equation*}
    \nabla \wt U(\theta) = - \frac{\abs\D}{\abs{\wt\D}}\sum_{x \in \wt\D} \nabla \log p(x \mid \theta) - \nabla \log p(\theta)
\end{equation*}
Appealing to the Central Limit Theorem, we imagine that the noisy estimate $\nabla \wt U(\theta)$ is normally distributed about $\nabla U(\theta)$, with some covariance matrix $V(\theta)$. The naïve adaptation of HMC to this stochastic scenario simply replaces $\nabla U(\theta)$ with $\nabla \wt U(\theta)$ in \cref{alg:hmc}. The corresponding discrete system is then the $\ep$-discretisation of the following dynamics:
\begin{align*}
    \dif\theta &= M^{-1}r \dif t \\
    \dif r &= - \nabla U(\theta) \dif t + \mc N(0, 2B(\theta) \dif t)
\end{align*}
where $B(\theta) = \frac 1 2 \ep V(\theta)$.

Unfortunately, such a dynamical system can diverge quite rapidly from the true posterior distribution \cite{neal-hmc}, which necessitates frequent Metropolis-Hastings steps. Such steps are costly since calculating the acceptance probability requires the use of the whole dataset. The method `Stochastic Gradient Hamiltonian Monte Carlo' (SGHMC) proposed in \cite{sghmc} addresses this shortcoming. The idea is to incorporate friction into the dynamical system, which works to counteract the noise introduced by selecting a subset of the data.


\subsection{SGHMC}

The SGHMC method adds a `friction term' $B M^{-1} r$ to the momentum update. Since we are unlikely to know the noise model $B$ in practice, we instead take an estimate $\wh B$ of $B$, together with a user-specified friction term $C \succeq \wh B$ and simulate the following dynamics.
\begin{align*}
    \dif\theta &= M^{-1}r \dif t \\
    \dif r &= - \nabla U(\theta) \dif t - CM^{-1}r \dif t + \mc N(0, 2(C - \wh B(\theta))\dif t) + \mc N(0, 2B(\theta) \dif t)
\end{align*}
The algorithm is given in \cref{alg:sghmc}. Ideally we would have $\wh B = B$, allowing us to reach the posterior distribution in an efficient manner. In practice, we must rely on an inaccurate estimate $\wh B$. The simplest choice is $\wh B = 0$. A better but more costly estimate is $\wh B(\theta) = \frac 1 2 \ep \wh V(\theta)$, where $\wh V$ is the observed (empirical Fisher) information \cite{sgld-fisher}. The friction term $C$ can then be taken as a hyperparameter, and set so as to counteract the inaccuracies of the estimate $\wh B$.

\begin{algorithm}
    \caption{The SGHMC algorithm}\label{alg:sghmc}
    \begin{algorithmic}
        \For{$t = 1,2, \ldots$}
            \State $r \sim \mc N(0,1)$ \Comment{Resample momentum}
            \For{$i = 1$ to $m$}
                \State $\theta \gets \theta + \ep M^{-1}r$
                \State $r \gets r - \ep \nabla \wt U(\theta) - \ep CM^{-1}r + \mc N(0, 2(C- \wh B(\theta))\ep)$
            \EndFor
        \EndFor
    \end{algorithmic}
\end{algorithm}

Lastly, we note that the added friction term prevents the dynamical system from diverging, meaning that we no longer require the Metropolis-Hasting's steps that were required for naïve SGHMC. This means we do not need to consider all of the data at each step in the SGHMC algorithm, making it fast to run even on large datasets.